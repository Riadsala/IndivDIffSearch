\documentclass[a4paper, oneside, 11pt, onecolumn]{article}
\usepackage[T1]{fontenc}
\usepackage{newtxtext,newtxmath}
\usepackage{graphicx}
\usepackage{subfigure}
\usepackage[round, comma, sort&compress, longnamesfirst]{natbib} 
% \usepackage[left=2cm,top=3cm,right=1.5cm,bottom=2cm,bindingoffset=0.5cm]{geometry}
\usepackage{enumerate}
\usepackage{fancyhdr}
\usepackage[title,titletoc,toc]{appendix}
\pagestyle{fancy}
\fancyhead{}
\fancyfoot{}
\fancyfoot[L]{\textit{running footer}}
 \fancyhead[L]{\textit{running title}}
\renewcommand{\headrulewidth}{0.8pt}
\renewcommand{\footrulewidth}{0.8pt}
\setlength\headheight{14pt}
\fancyfoot[RO] {\thepage}
\usepackage{authblk}
%%%%%%%%%%%%

\begin{document}

\title{Individual Differences Across Visual Search Tasks}

\author{A. D. F. Clarke, J. L. Irons, A. B. Leber and A. R. Hunt}
\affil{Aberdeen, Essex and Ohio}

\maketitle

\begin{abstract}
Some abstract goes here
\end{abstract}

%%%%%%%%%%%%%%%%%%%%%%%%%%%%
\section{Introduction}
%%%%%%%%%%%%%%%%%%%%%%%%%%%%


%%%%%%%%%%%%%%%%%%%%%%%%%%%%
\section{Methods}
%%%%%%%%%%%%%%%%%%%%%%%%%%%%


\subsection{Participants}
How many? Will all testing be done at the University of Aberdeen? How do we justify our sample size (a question commonly asked by journals now!)?

\subsection{Materials and Procedures}

The study consists of three different paradigms from the visual search literature in which strong individual differences were found \citep{nowakowsak2017, irons-leber2016, kristjansson2014}.
{}
\subsubsection{A: Split-half array search}

\subsubsection{B: Attentional Control}

\subsubsection{C: Conjunction Foraging}

\subsection{Planned Analysis}

\subsubsection{A: Split-half array search}

In order to characterise an individual's behaviour in this task, we will compute the proportion of the first $n$ fixations that were on heterogeneous (difficult) side of the stimuli, over all target absent trials\footnote{Only take correct trials?}. \cite{nowakowsak2017} demonstrated a strong correlation between an this metric (for $n=5$) and reaction times ($r=$). However, a re-analysis of their data shows that an even stronger correlation is obtained with $n=3$ (see Figure \ref{fig:nowakowskaBestN})

\begin{figure}
\centering
\subfigure[][]{\includegraphics[width=6cm]{../NowakowskaClarkeHunt2017/scripts/r_by_nfix.pdf}}
\subfigure[][]{\includegraphics[width=6cm]{../NowakowskaClarkeHunt2017/scripts/best_r_scatter.pdf}}
\caption{Selecting the best $n$.}
\label{fig:nowakowskaBestN}
\end{figure}

\subsubsection{B: Attentional Control}

Following \cite{irons-leber2016}, individual differences will be characterised using two measures: 
\begin{itemize}
\item Proportion of optimal choices on plateaus (i.e., where the distractor colour is . An optimal choice is defined as responding to whichever of the two target colours has the fewest items in the display. This will be based on correct plateau trials only. \\
\item Switching frequency. Switching is defined as responding to a different target color on trial $N+1$ than on trial $N$, and is presented as a proportion of the total number of trials (excluded incorrect trials, the trial after an incorrect trial, and the first trial of each block).\\
\end{itemize}

Both measures have been shown to correlate with overall RT in previous experiments (Proportion Optimal $r =-.56$, $p < .001$; Switch Frequency $r = .45$, $p = .001$; based on $N = 50$ and using the same number of trials as in the current experiment). As additional validation, correctional analyses between both measures and RT will also be conducted in the current experiment.

\subsubsection{C: Conjunction Foraging}

Data analysis will follow \cite{kristjansson2014,johannesson2016}. The main measure of interest will be the average run length per trial. A run is defined as a succession of one or more of the same target type which is followed and preceded by the other target type or no target. The average length is the average number of consecutive target selections per run. As with the other experiments, we will also measure the average response (time to select the final target on correct trials).


\subsubsection{Correlations}

We will present similarity matrices for how well the reaction times of one study correlate with the others. We will then go on to explore the correlations between the various strategy features outlined above. 

We will use Bayesian analysis to estimate all effect sizes. This avoids the need to correct $p$-values for multiple comparisons as is required under null hypothesis significance testing. Analysis will be carried out with several different choices of prior to investigate robustness (ie, uninformative/uniform [-1,1] prior and a Normal(0,1) prior). If a variable is highly skewed, it will be log-transformed. All variables will be $z$-scaled. 


\subsection{Exploratory Analysis}

We will carry out additional analysis, above and beyond what has been documented above, but the exact nature of this will be contingent on the nature of the results. Something like PCA may be interesting. 

We will also explore modeling the non-linear relationships between the variables if appropriate, as there is no reason to assume that the relationships must be linear. 

%%%%%%%%%%%%%%%%%%%%%%%%%%%%
\section{Results}
%%%%%%%%%%%%%%%%%%%%%%%%%%%%

%%%%%%%%%%%%%%%%%%%%%%%%%%%% 
\section{Discussion}
%%%%%%%%%%%%%%%%%%%%%%%%%%%%


%%%%%%%%%%%%%%%%%%%%%%%%%%%%
\begin{appendices}
\section{Hetero-Homo-geneous Array Search}
%%%%%%%%%%%%%%%%%%%%%%%%%%%%

%%%%%%%%%%%%%%%%%%%%%%%%%%%%
\section{Attentional Control Settings}
%%%%%%%%%%%%%%%%%%%%%%%%%%%%

%%%%%%%%%%%%%%%%%%%%%%%%%%%%
\section{Conjunction Foraging}
%%%%%%%%%%%%%%%%%%%%%%%%%%%%
\end{appendices}

\bibliographystyle{plainnat}
\bibliography{literature}

\end{document}




\RequirePackage{fix-cm}
\documentclass[]{rsos}%%%%where rsos is the template name

%% *** Do not adjust lengths that control margins, column widths, etc. ***
\usepackage{subfigure}
\usepackage{lineno}
\begin{document}

\linenumbers
%%%% Article title to be placed here
\title{Insert the article title here}

\author{%%%% Author details
Alasdair D. F. Clarke$^1$, Jessica L Irons$^2$, Warren James$^3$, Andrew B Leber$^2$ and Amelia R Hunt$^3$}

%%%%%%%%% Insert author address here
\address{$^{1}$Department of Psychology, University of Essex, Colchester, UK\\
$^{2}$Department of Psychology, The Ohio State University, Columbus, USA\\
$^{3}$School of Psychology, University of Aberdeen, Aberdeen, UK
}

%%%% Subject entries to be placed here %%%%
\subject{Behaviour, evolution}

%%%% Keyword entries to be placed here %%%%
\keywords{visual search, optimal behaviour,
eye movements}

%%%% Insert corresponding author and its email address}
\corres{Alasdair Clarke\\
\email{a.clarke@essex.ac.uk}}

\begin{abstract}
Some abstract goes here
\end{abstract}

%%%%%%%%%% Insert the texts which can accomdate on firstpage in the tag "fmtext" %%%%%

\begin{fmtext}
%%%%%%%%%%%%%%%%%%%%%%%%%%%%
\section{Introduction}
%%%%%%%%%%%%%%%%%%%%%%%%%%%%

As is common in cognitive psychology, most visual search literature has focused on how the average participant performs in the task, despite it being well known that there is a great deal of variability between one subject and the next. These differences can emerge from several different sources of variation: tiredness\cite{mackworth1948}, information-processing ability, speed-accuracy trade-off, motivation, visual impairments \cite{nowakowska2016}, and search strategies\cite{boot2006}. Although their existence has previously been noted \cite{mackworth1948}, a rigorous examination of individual differences in visual search has been largely ignored and questions about their importance and stability remain relatively under explored. 

In the current study, we focus on one source of individual differences in visual search: strategy. By search strategies, we refer to a collection of behaviours that all observers can freely choose from when they search a display. Examples include choosing to adopt a systematic left-to-right and top-to-bottom strategy\cite{gilchrist2006}, or the type of guided search behaviour in which locations that are more likely to contain the target are prioritised \cite{wolfe2015}. Perhaps surprisingly, previous literature has shown that both ideal observer \cite{najemnik-geisler2008} and stochastic \cite{clarke2016} models can explain the number of fixations required to locate a target in noise. These apparent contradictory results could potentially explained by large individual differences \cite{nowakowska2017}. 

\end{fmtext}
\maketitle

A striking example of the effect of strategy is given by \cite{boot2006}. They asked participants to monitor a cluttered display for an object changing colour or object onset. Large individual differences were found with respect to the number of saccades participants made while monitoring the stimulus, and this was negatively correlated with detection performance. 

\cite{proulx2011} demonstrated that the different weighting between top-down and bottom-up factors in a conjunction search task explained \ldots Interestingly, they could that around a third of participants were unaware of the search strategy they employed. 


In our own work, we have previously shown that there are large differences between individuals in terms of the search strategies used to find a target among distracters. The Adaptive Choice Visual Search (ACVS) paradigm\cite{irons-leber2016} was used to examine strategies in the use of feature-based attention. [See below and supplementary materials for a description of this paradigm.] Participants varied substantially along two key dimensions: how frequently they used the more effective target feature to search (varying from chance performance to near optimal), and how often they changed search features. Further work has shown that these differences are stable over time (between one and ten days) with test-retest correlations of around $r = 0.8$\cite{irons-leber2018}.

Using a different paradigm\cite{nowakowsak2017}, we aimed to discriminate between optimal\cite{najemnik-geisler2008} and stochastic \cite{clarke2016} search strategies in a simple visual search task involving arrays of line segments (see Figure \ref{fig:exampleStimuli}). The line segments were arranged such that those on one side of the display all had a very similar orientation, while those on the other side had higher variance. This was done so that if the target appeared on the homogeneous side, it would be highly salient. Trials in which the target appeared on the heterogeneous side of the display would be much harder. With these stimuli, the optimal eye movement strategy is to only search the heterogeneous half of the stimulus, as targets on the homogeneous side can be detected with peripheral vision. We found that while some participants initially searched the displays near optimally, others carried out strategies counter to this, failing to even match the performance of the stochastic searcher. Furthermore, the degree to which they made saccades in line with the optimal search strategy was strongly correlated with their reaction times. 

\begin{figure}
\centering
\subfigure[][]{\includegraphics[height=3cm]{figures/split-half.png}}
\subfigure[][]{\includegraphics[height=3cm]{figures/adaptive.png}}
\subfigure[][]{\includegraphics[height=3cm]{figures/foraging.png}}
\caption{Example stimulus from the (a) \textit{split-half}, (b) \textit{adaptive choice} and (c) \textit{foraging} paradigms}
\label{fig:exampleStimuli}
\end{figure}

Another example of differences in search strategy comes from the foraging literature \cite{kristjansson2014,johannesson2016}. In this context, foraging is a visual search task in which there are multiple targets on each trial. Participants were asked to search through a set of items from four categories, two of which were classed as targets. In the conjunction condition (i.e. searching for red-horizontal and green-vertical line segments among red-vertical and green-horizontal distracters), most observers searched in runs of one target category or another. This strategy has previously been observers in animal foraging literature \cite{dawkins1971}. However, a sub-set of observers, termed ``super-searchers'' showed no switch cost. 

A common theme emerging from the studies in the observation that individual strategies vary in their degree of effectiveness or optimality. However, ``visual search'' encompasses a wide range of specific tasks, each tapping into a different aspect of behaviour (feature-based attention, [spatial/fixation patterns??]). The aim of the present study is to investigate the extent to which individual differences are stable across different visual search paradigms. Are observers who use the optimal strategy in the split-half search arrays also more optimal in the ACVS task? Does it make sense to talk about `super-searchers' who show above average performance in a range of search tasks? Are the super-foragers consistently better or worse than more typical searchers in the other two paradigms? As a secondary question, we will measure the test-retest reliability of the differences found in the split-half array paradigm. 


%%%%%%%%%%%%%%%%%%%%%%%%%%%%
\section{Methods}
%%%%%%%%%%%%%%%%%%%%%%%%%%%%

\subsection{Participants}
We aim to find 64 participants to volunteer to take part in this experiment. Participants will be students from the University of Aberdeen. Some will be compensated with course credit and some will be paid \pounds 15 for their time. Sample size was determined in part due to constraints with counter-balancing; there are 16 different possible orders of tasks/conditions; we will run four participants in each order for a total of 64. All participants will sign a form giving informed consent. The study has already been approved by the University of Aberdeen Psychology Ethics Committee.

A sample of 64 participants means we should be able to detect a correlations with $r > 0.342$ with $\alpha = 0.05$, $\beta = 0.80$ between the different visual search paradigms. Given the nature of our results, we see no need to apply a conservative correction for multiple comparisons. 

\subsection{Materials and Procedures}

The study consists of three different paradigms from the visual search literature in which large individual differences were found \cite{nowakowsak2017, irons-leber2016, kristjansson2014}. Example stimuli can be seen in Figure \ref{fig:exampleStimuli}.



The display was presented on a 17-inch CRT monitor with a resolution of $1400 \times 
1050$. However, due to a miscommunication, some participants (\textbf{get exact numbers from script}) were run at $1024 \times 768$ or $1600 \times 1200$. Stimulus generation, presentation and data collection were controlled by MATLAB and psychophysics toolbox \cite{brainard1997} run on a Powermac. 

A brief overview of each paradigm is given below, with full details in \textit{supplementary materials}. 

\subsubsection{Split-half Array Search}

Stimuli consisted of arrays of black oriented line segments against a grey background. Each line segment had a length of $\approx$$1.6^{\circ}$ of visual angle irrespective of screen resolution. The target was oriented $45^{\circ}$ clockwise, while the distractor items had a random orientation with a mean of $45^{\circ}$ anti-clockwise. The variance was low ($18^{\circ}$) on one half of the display to create a homogeneous texture, and high ($95^{\circ}$) on the other side to create a heterogeneous texture. This means that when the target is present on the homogeneous side of the stimulus, it can be easily be detected with peripheral vision, but when it is in the heterogeneous half, it is much harder to detect. There were a total of 160 trials and homo- and heterogeneous sides of the display were randomly varied from trial to trial. The position of the dominant eye was recorded using a desktop-mounted EyeLink 1000 eye tracker (SR Research, Canada). 

This paradigm was carried out twice to give us an estimate of how consistent participants are in their search strategy over time. The two sessions were identical.

\subsubsection{Adaptive Choice Visual Search}

The ACVS was based on the task described in \cite{irons-leber2018}, Experiment 1.

% These are the sizes of the boxes for each resolution, I'm not sure how best to incorporate this into the text though... here's a table for the meantime;
% Thanks! Stim size is not critical for our experiment, so it doesn't matter too much if it's too complicated (just habit to have it in there :)
\begin{table*}
	\centering
	\small
	\begin{tabular}{cc|c}
	    Experiment	    &	Resolution	     & Size of stim \\
		\hline
		Adaptive Choice & $1024 \times 768$  & $\approx$$1.8^{\circ}$ \\
		Adaptive Choice & $1400 \times 1050$ & $\approx$$1.3^{\circ}$ \\
		Adaptive Choice & $1600 \times 1200$ & $\approx$$1.1^{\circ}$ \\
		Foraging        & $1024 \times 768$  & $\approx$$1^{\circ}$ \\
		Foraging        & $1400 \times 1050$ & $\approx$$0.7^{\circ}$ \\
		Foraging        & $1600 \times 1200$ & $\approx$$0.6^{\circ}$ \\
	\end{tabular}
	\caption{Size of stimuli for each experiment and screen resolution}
	\label{tab:Screen_Res}
\end{table*}

% Let me know if you want me to shift some of this into the Supp
Each search display was composed of 54 small squares arranged in three concentric rings around fixation, with 12, 18 and 24 items in the inner, middle and outer rings respectively. The same screen was used as in the Split-half Array Search, however, due to changes of screen resolution, the size of the squares changed. Participants, were sat $\approx$47cm from the screen. Of the 54 squares, 13 were red, 13 were blue, 14 were green and 14 were "variable". Variable distractors change colours from trial-to-trial according to a 24 trial cyclical pattern: the distractors would be red for 5 trials (called a "red plateau"), then across a period of 7 trials, they would change colour from almost red to magenta (at the fourth trial in the transition) to almost blue. The variable distractor would then be blue for 5 trials (blue plateau), and then transition back from almost blue through magenta to almost red. 

A white digit appeared inside each square. Two targets - a red square and a blue square each with a digit between 2 and 5 - were embedded in every search display. The two target digits were always different, to enable us to distinguish the chosen target. The remaining red, blue and variable squares all contained digits between 6-9. Green squares could contain any digit between 2-9. The location of the targets and distractor within the search display were randomized on each trial.

Participants were informed that the search displays would contain two targets on every trial, that they need only find one target on each trial and that they were always free to search for either one.   

\subsubsection{Conjunction Foraging}

The foraging task was based on \cite{kristjansson2014} and \cite{johannesson2016}. Participants completed the feature foraging and conjunction foraging tasks on separate days, with the order counterbalanced (was it counterbalanced?).

In the feature foraging task, search displays contained 80 small circles, 20 red, 20 green, 20 blue  and 20 yellow. Stimuli were arranged in a 10 x 8 grid, but the position of each item within the grid space was jittered to create a more random spatial arrangement. The location of item colours to grid locations was completely randomized. 

For half of the participants, targets were red and green circles, and for the other half of participants, targets were blue and yellow circles. Participants were asked to collect all of the targets within a trial by using the mouse to click on each target. Clicking on a target caused it to disappear from the display. If the participant clicked erroneously on a non-target, the trial was immediately ended and a replacement trial was begun. 

In the conjunction foraging task, search displays were composed of both circles and squares. For half of the participants, the shapes were red and green (equal numbers of red circles, red squares, green circles and green squares), and for the remaining participants the shapes were blue and yellow. Targets were defined by conjunctions of colour and shape (e.g., red squares and green circles, with red circles and green squares as distractors). The assignment of targets and distractors was assigned at random for each participant. The procedure was otherwise identical to the feature foraging task. 

%% For the assignment of distractors and targets, we generated a two lists of random numbers (1-2 for easy search, and 1-4 for hard search). Pretty sure we did that because counterbalancing became very tricky once you accounted for that as well. 

\subsection{Planned Analysis}

% Structure-wise, would it be easier for the reader to have methods and planned analyses nested under each task?

\subsubsection{Split-half array search}

In order to characterise an individual's behaviour in this task, we will compute the proportion of the first $n$ fixations that were on heterogeneous (difficult) side of the stimuli, over all target absent trials\footnote{Only take correct trials?}. \cite{nowakowsak2017} demonstrated a strong correlation between an this metric (for $n=5$) and reaction times ($r=.53$). However, a re-analysis of their data shows that an even stronger correlation is obtained with $n=3$.

\subsubsection{Attentional Control}

Participants with accuracy more than 3 SD below the group mean were excluded from analyses. For RT analyses, trials with RTs less than 300ms or more than 3 SD about the participant's mean were excluded. 

Two measures of individual strategy use were used: 1) Optimal choices, defined as percent of plateau trials in which the individual chose the optimal target (i.e., the target with the fewest distractors. When the variable distractor was red, the optimal choice was blue, and vice versa), and 2) Switch rate, the percent of trials in which the individual switched target colour (i.e., the colour chosen on trial N was different to the colour chosen on trial N-1).  

\subsubsection{Conjunction Foraging}

Only completed, accurate trials were analysed. RTs were defined across the entire trial (i.e., from the start of the trial until the final target was collected). The main measure of interest was average run length per trial. A run was defined as a succession of one or more of the same target type, which was followed and preceded by the other target or no target. The average run length was the average number of target selections in a run. 

\subsection{Exploratory Analysis}

We will carry out additional analysis, above and beyond what has been documented above, but the exact nature of this will be contingent on the nature of the results. Something like PCA may be interesting. 

%%%%%%%%%%%%%%%%%%%%%%%%%%%%
\section{Results}
%%%%%%%%%%%%%%%%%%%%%%%%%%%%

\subsection{Replication of each paradigm}

\subsubsection{Split-half Array Search}
Our results are broadly in line with \cite{nowakowsak2017}. The correlation between accuracy and reaction times between the two sessions is shown in Figure \ref{fig:splithalf_summary}(a, b). We can clearly see that there are large differences from one participant to the next in terms of both the proportion of hard targets found, and reaction times. Furthermore, test-retest reliability appears to be reasonable, with Pearson's $r \in [0.65-0.86]$ ($95\%$ confidence interval) for accuracy in finding targets on the hard heterogeneous half of the display. We get similar scores for the correlation between sessions \textit{a} and \textit{b} for heterogeneous targets, ($r \in [0.66-0.87]$), homogeneous targets ($r \in [0.56-0.82]$) and target absent ($r \in [0.65-0.86]$). The reduced correlation for the homogeneous targets is likely due to the restricted range.

\begin{figure}
\centering
\subfigure[][]{\includegraphics[width=3.25cm]{../Scripts/lineseg/scratch/acc_correlation.pdf}}
\subfigure[][]{\includegraphics[width=3.25cm]{../Scripts/lineseg/scratch/rt_correlation.pdf}}
\subfigure[][]{\includegraphics[width=3.25cm]{../Scripts/lineseg/scratch/strat_corr.pdf}}
\subfigure[][]{\includegraphics[width=3.25cm]{../Scripts/lineseg/scratch/strat_compare_meanlog_rt.pdf}}
\caption{Each point represents a participant and the error-bars indicate 95\% confidence intervals. Correlation between the two sessions of the \textit{split-half} paradigm for (a)  accuracy (TP-heterogeneous trials only); (b) reaction times and (c) search strategy (TA trials only). (d) Initial search strategy correlates with reaction times in both sessions. }
\label{fig:splithalf_summary}
\end{figure}

We can also look at the initial search strategies adopted by our participants \ref{fig:splithalf_summary}(c, d). Again, we see large and stable individual differences across the two sessions (test-retest $r \in [0.57, 0.83]$ for the proportion of the first five saccades to the heterogeneous half of the display for target absent trials). More importantly, as with \cite{nowakowsak2017}, we see that the search strategies give a good correlation with reaction times in both session a, $r \in [0.56, 0.84]$ and session b, $r \in [0.50, 0.81]$.


\subsubsection{Adaptive Choice}

The results for the ACVS were consistent with our previous findings \cite{irons-leber2016,irons-leber2018} (see Figure 3?). %Figure 3 (? AC_OptCh_x_AllSw) shows that there were wide individual differences in both percent option (range $33.62\%$ – $97.50\%$, $M = 56.99$, $SD = 14.75$) and switch rate (range = $0\% - 56.68\%$, $M = 28.46$, $SD = 13.54$). Although we did not assess it here, good test-retest reliability has been found previously for both percent optimal ($r = .83$) and switch rate ($r = .77$) \cite{irons-leber,2018}.

\subsubsection{Conjunction Foraging}

The multiple-target foraging results were also in line previous findings \cite{kristjansson2014,johannesson2016}. As expected, run lengths were shorter for feature foraging ($M = 3.24$, $SD = 3.19$) than conjunction foraging ($M = 10.64$, $SD = 7.01$), $t(55) = 8.03$, $p < .001$ (PREFER DIFFERENT STATS HERE? CI [-9.25,-5.55]) suggesting more frequent foraging for multiple targets concurrently when those targets were defined by features than by conjunctions. However, both showed significant individual variation: Feature run length varied between 1.87 and 20, and conjunction run length varied between 1.00 and 19.67. Figure 4 (?) depicts run length across the trials separately for each individual. Twenty ``super-searchers'' were revealed, identified by short run lengths for both feature and conjunction targets [THAT NUMBER IS JUST BASED ON EYE-BALLING THE FIGURE, BUT WE COULD COME UP WITH SOME SORT OF PRECIOUS CUT OFF – WHAT DO YOU THINK?]. 


\subsection{Correlations Between Paradigms}

The results above demonstrate that we have successfully replicated the previous findings around individual differences in visual search strategy. As an initial sweep, we look at overall performance in terms of RT (note that this can be influenced not only by strategy, but also by individuals' abilities, such as their information-processing speed and their ability to ignore distractors). Are individuals who are fast on one visual search task fast on another? In all comparisons the correlations are weak, typically $0.27 < r <0.30$ but positive (Figure \ref{fig:between_para_rt}). None of these correlations are statistically significant ($p>0.05$). Even if we optimistically take all the data together as suggesting a robust correlation in reaction times from paradigm to paradigm, the upper bound of $r=0.30$ this correlation only accounts for at best $10\%$\footnote{i.e., $R^2 = 0.3^2 = 0.09$} an individual's performance. 

\begin{figure}
\centering
\subfigure[][]{\includegraphics[width=3.25cm]{../Scripts/scratch/ls_v_ac_mean_log2rt.pdf}}
\subfigure[][]{\includegraphics[width=3.25cm]{../Scripts/scratch/ls_v_ac_opt.pdf}}
\caption{Correlation between the split-half and adaptive choice paradigms for (a) reaction times and (b) optimal behaviour.}
\label{fig:between_para_rt}
\end{figure}

[I WONDER IF THIS SHOULD GO BEFORE THE RT CORRELATIONS? SINCE AS YOU SAY, THIS IS THE MAIN AIM OF THE PAPER. THEN RT CORRELATIONS COULD GO AS A SECONDARY AIM] Given the low correlations between reaction times, it seems unlikely that we will find that individuals who search efficiently and optimally in one paradigm will search well in another (the original motivation for our study) [I WOULD BE INCLINED NOT TO SAY THIS, BECAUSE IT SEEMS TO MINIMIZE THE PURPOSE OF LOOKING AT STRATEGY AT ALL. IE IT SUGGESTS THAT ALL OF THE VARIATION IN STRATEGY COULD BE ENCAPSULATED BY RT. BUT THERE ARE OTHER FACTORS THAT ALSO INFLUENCE RT]. The analysis supports this hypothesis. For example, the correlation between the proportion of fixations to the heterogeneous side of the display in the split-half paradigm, and proportion of optimal targets found in the adaptive choice task is $r=-0.03$. Further results are presented in supplementary materials. 

%%%%%%%%%%%%%%%%%%%%%%%%%%%%
\section{Discussion}
%%%%%%%%%%%%%%%%%%%%%%%%%%%%

The results presented above are somewhat surprising. 


I think he sees it as a positive thing that they tasks don't correlate, because it suggests they capture unique variation in behaviour
glass half full approach 

Mention \cite{vogel2008}.
From \cite{proulx2011} - \textit{Vogel and colleagues have found that an individual’s ability to remember a greater number of items using working memory is related to a filtering capacity in visual search that suppresses attentional capture by distracting visual information [10,11,12]. Behavioral work in visual search has extended this research to
demonstrate that working memory correlates only with top-down
visual search performance where task relevance is crucial, but not
with  bottom-up  visual  search  tasks  where  salience  in  the
environment guides attention \cite{sobel2007}}


Mention \cite{stoet2011}.


\bibliographystyle{plain}
\bibliography{literature}

\end{document}


